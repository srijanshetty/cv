%----------------------------------------------------------------------------------------
%	PACKAGES AND OTHER DOCUMENT CONFIGURATIONS
%----------------------------------------------------------------------------------------

\documentclass[a4paper,10pt]{article} % Default font size and paper size

\usepackage{fontspec} % For loading fonts
\defaultfontfeatures{Mapping=tex-text}
\setmainfont[SmallCapsFont = Fontin SmallCaps]{Fontin} % Main document font

\usepackage{xunicode,xltxtra,url,parskip} % Formatting packages

\usepackage{marvosym} % Allows the use of symbols
\usepackage{scrpage2} % Provides headers and footers configuration


\usepackage[usenames,dvipsnames]{xcolor} % Required for specifying custom colors

\usepackage[big]{layaureo} % Margin formatting of the A4 page, an alternative to layaureo can be \usepackage{fullpage}
% To reduce the height of the top margin uncomment: \addtolength{\voffset}{-1.3cm}

\usepackage{hyperref} % Required for adding links	and customizing them
\definecolor{linkcolour}{rgb}{0,0.2,0.6} % Link color
\hypersetup{colorlinks,breaklinks,urlcolor=linkcolour,linkcolor=linkcolour} % Set link colors throughout the document

\usepackage{tabularx,colortbl} % Advanced table configurations

\usepackage{titlesec} % Used to customize the \section command
\titleformat{\section}{\Large\scshape\raggedright}{}{0em}{}[\titlerule] % Text formatting of sections
\titlespacing{\section}{0pt}{3pt}{3pt} % Spacing around sections

\begin{document}

\font\fb=''[cmr10]'' % Change the font of the \LaTeX command under the skills section

\titleformat{\section}{\large\scshape\raggedright}{}{0em}{}[\titlerule] % Section formatting

\pagestyle{scrheadings} % Print the headers and footers on all pages

\addtolength{\voffset}{-0.5in} % Adjust the vertical offset - less whitespace at the top of the page
\addtolength{\textheight}{3cm} % Adjust the text height - less whitespace at the bottom of the page

%----------------------------------------------------------------------------------------
% FOOTER SECTION
%----------------------------------------------------------------------------------------

\renewcommand{\headfont}{\normalfont\rmfamily\scshape} % Font settings for footer

\cofoot{
\addfontfeature{LetterSpace=20.0}\fontsize{10.5}{17}\selectfont % Letter spacing and font size

I-308, IIT Kanpur {\large\textperiodcentered} Kanpur {\large\textperiodcentered} UP 208016\\ % Your mailing address
{\Large\Letter} srijan.shetty@gmail.com \ {\Large\Telefon} +(91) 9005900383 % Your email address and phone number
}

%----------------------------------------------------------------------------------------
% HEADER SECTION
%----------------------------------------------------------------------------------------

{\addfontfeature{LetterSpace=20.0}\fontsize{36}{36}\selectfont\scshape Srijan R. Shetty} %Your name at the top
\\ \\
{Senior Undergraduate \\%
Department of Computer Science and Engineering \\%
Indian Institute of Technology, Kanpur}

%----------------------------------------------------------------------------------------
%   INTERESTS
%----------------------------------------------------------------------------------------
\section{Interests}

Web Development, Natural Language Processing, Programming Languages, Distibuted Systems,
Networks

%----------------------------------------------------------------------------------------
%	EDUCATION
%----------------------------------------------------------------------------------------
\section{Education}

\begin{tabular}{rl}

    \textsc{Current} & B. Tech in \textsc{Computer Science and Engineering},\\
                     & \textbf{Indian Institute of Technology}, Kanpur\\
                     & \normalsize \textsc{Gpa}: 9.6/10.0 \\
                     & \textbf{Relevant Courses}: \\
                     & Computer Organization (CS220) \\
                    & Operating Systems (CS330) \\
                    & Compilers (CS335) \\
                    & Computer Networks (CS425) \\
                    & Theory of Computation (CS3) \\
                    & Computing Laboratory (CS251 \& CS252)\\
                    & Discreet Mathematics (CS201)\\
                    & Logic for Computer Science (CS202A)\\
                    & Data Structure and Algorithms (CS210)\\
                    & Design and Analysis of Algorithms (CS345)\\
                    & Principles of Programming Languages (CS350)\\
                    & Computer and Internet Security (CS628)\\
                    & Artificial Intelligence (CS365)\\
                    & Abstract Algebra (CS202B)\\
                    & Linear Algebra (MTH102)\\
                    & Computational Methods in Engineering (ESO208A)\\
                    & Multivariate Calculus (MTH101)\\
                    & Technical Communication(CS300)\\
                     \\

    \textsc{July} 2011 & \textsc{CBSE} Board, \normalsize\textbf{The Emerald Heights International School}, Indore\\
                       &\normalsize \textsc{Percentage}: 91.4 \\
                       \\

    \textsc{July} 2011 & \textsc{ICSE} Board,  \normalsize\textbf{The Laurels School International}, Indore \\
                       & \normalsize \textsc{Percentage}: 95.4 \\
                       \\

\end{tabular}

%----------------------------------------------------------------------------------------
%	SCHOLARSHIPS AND CONFERENCES
%----------------------------------------------------------------------------------------
\section{Scholarships and Conferences}

\begin{tabular}{rl}

    \textsc{Jan} 2011 & Kishore Vaigyanik Protsahan Yojana (KVPY) Fellowship \\
    \textsc{Dec} 2008 & FIITJEE Talent Reward Examination Scholarship \\
    \textsc{Dec} 2009 & CSIR Programme on Youth for Leadership in Science (CPYLS camp) \\

\end{tabular}

%----------------------------------------------------------------------------------------
%   WORK EXPERIENCE
%----------------------------------------------------------------------------------------

\section{Work Experience}

\begin{tabular}{r|p{11cm}}
    \textsc{May-Jul 2014} & Research Intern at \textsc{Microsoft Research India}, Bangalore\emph{}\\
        & \footnotesize{Worked on Distibuted Systems} \\

    \textsc{May-Jul 2013} & Software Development at \textsc{Aurus Networks}, Bangalore\emph{}\\
        & \footnotesize{Used bleeding edge web technologies \textbf{HTML5, javascript, CSS3, Flash and XMPP} to develop a
            \textbf{technology-agnostic video player and real time communication/interaction services.}
            for the company's MOOC Service \textbf{CourseHub}}

\end{tabular}

%----------------------------------------------------------------------------------------
%  COURSE PROJECTS
%----------------------------------------------------------------------------------------
\section{Course Projects}

\begin{tabular}{r|p{11cm}}

    \textsc{Spring 2014} & JavaScript to MIPS Compiler, \textsc{\raggedright Compilers} \\
        & \footnotesize{An end to end compiler for a version of ECMAScript5.1 with static type hinting,
            first class functions, and runtime support was implemented in Python which compiled to MIPS
            architecture.}\\
            \\

    \textsc{Spring 2014} & Hindi author attribution, \textsc{\raggedright Artificial Intelligence} \\
        & \footnotesize{The first half of this project was a multi-class clustering problem, wherein authors were to be clustered
            according to their writing style. The features used for the same were, word frequency counts and collocation frequencies.
            The second half of the project was to classify the text snippets into different authors using supervised learning techniques.
            The project was supervised by \textbf{Professor Amitabha Mukherjee} and was inherently difficult due to the unavailability
            of stemmers and POS taggers for Hindi.} \\
            \\

    \textsc{Fall 2013} & NachOS, \textsc{Operating Systems} \\
        & \footnotesize{In the three phases of this project undertaken in the aegis of \textbf{Professor Mainak Chaudhuri},
            three different areas of an Operating System were explored.
            The first phase involved implementation of system calls; the second was process management and process scheduling;
            and the last was different memory management schemes and page replacement algorithms }\\
            \\

    \textsc{Fall 2013} & Peer to Peer File transfer, \textsc{Computer Networks} \\
        & \footnotesize{To understand the working of Transport Layer Protocols and socket programming
            , a peer to peer file transfer architecture,
            mimicking that of BitTorrent was created in Node.js to allow easy file sharing acorss any Local Area Network.
            The project was supervised by \textbf{Professor Dheeraj Sanghi}. \href{https://github.com/srijanshetty/nodesock} {Repo}}\\
            \\

    \textsc{Fall 2013} & IP Spoofing, \textsc{Computer Networks} \\
       & \footnotesize{Under the guidance of \textbf{Professor Dheeraj Sanghi}, different methodologies of spoofing
            IP addresses and different uses of spoofing IPs were implemented and tested in a secure environment.}\\
            \\

    \textsc{Fall 2013} & Erlang Traffic Server, \textsc{Principles of Programming Languages} \\
       & \footnotesize{A simple traffic Server was implemented in Erlang to grok message passing concurrency and other
        functional aspects of the language under the guidance of \textbf{Professor P. Kurur}}\\
            \\

    \textsc{Fall 2013} & Metro Travel Plan(Prolog), \textsc{Principles of Programming Languages} \\
       & \footnotesize{To grok the concepts of logic programming, we were tasked to implement a shortest path
        algorithm on Delhi Metro route using solely logic programming constructs. This project was also supervised by
        \textbf{Professor P. Kurur}}\\
            \\

    \textsc{Spring 2013} & 8-bit General Purpose Computer on a FPGA, \textsc{Computer Organization} \\
        & \footnotesize{The GPC had a load-store architecture and was written in System Verilog and implemented on a \textbf{Xilinx Spartan 3 FPGA}.
            The project was under the supervision of \textbf{Professor Subhajit Roy}.
            It had a limited, but powerful Instruction Set Architecture which could implement recursive functions, jumps,
            loops  other basic building blocks of a simple assembly language.
            \href{https://github.com/srijanshetty/220_y11} {Github Repo}} \\
            \\

    \textsc{Spring 2013} & Web Crawler, \textsc{Computing Laboratory} \\
        & \footnotesize{A python based web crawler which could obtain paper submission deadlines of a supplied conference name.
            BeautifulSoup and urllib along with simple pattern matching through regular expressions were used to implement crawling.
            \href{https://github.com/srijanshetty/crawler} {Github Repo} } \\
            \\

    \textsc{Fall 2011} & "No situation is unique and certain moral principles can be applied across all situation",
        \textsc{Introduction to Philosophy} \\
        & \footnotesize{ An argumentative presentation proving that certain moral principles can be applied throughout;
                by proving different situations have certain fundamental similarity.
                Albeit this would require a gross simplification of different situations.
                The reviewing instructor was \textbf{Professor Vineet Sahu}} \\

\end{tabular}

%----------------------------------------------------------------------------------------
%  SIDE PROJECTS
%----------------------------------------------------------------------------------------
\section{Side Projects}

\begin{tabular}{r|p{11cm}}

    \textsc{May-July 2012} & Voicing, Archiving and Networking the Information \textsc{(VANI)}, IIT Kanpur\\
        & \footnotesize{Vani is an \textbf{extensible} platform made for the campus community consisting of Wiki, Forums,
                Community search, Calendars etc.; built using \textbf{Drupal} under the mentorship of
                \textbf{Professor Manindra Agarwal} and \textbf{Professor T. V. Prabhakar} } \\
            \\

    \textsc{May-July 2012} & GNU/Linux Exploration, Programming Club, IIT Kanpur\\
        & \footnotesize{ The project involved exploration of various facets of \textbf{GNU/Linux} like the file system,
            process management, memory management, shell interface et al.
            It was essentially a theoretical run through GNU/Linux.
            \href{https://docs.google.com/document/d/1ZHO9w36aoq3oaZBR4Um1AOmDfiTDAEgM6baQAu3icw4/edit?usp=sharing} {Documentation} } \\
            \\

    \textsc{July 2013} & Web development\\
        & \footnotesize{Webpage of \textbf{Udghosh}, the annual sports festival of IIT Kanpur. \href{www.udghosh.org} {Webpage} } \\
        & \footnotesize{Homepage of \textbf{Hall 5}, IIT Kanpur. \href{http://www.iitk.ac.in/hall5} {Webpage}. } \\
        \\

    \textsc{July 2013} & Projects on Github \\
        & \footnotesize{\textbf{ShuffleRun}: a music player which changes tracks on the basis of running speed of the user
            \href{https://github.com/srijanshetty/ShuffleRun} {Repo} }\\
        & \footnotesize{\textbf{GYPH}: a music player which changes tracks on the basis of running speed of the user
            \href{https://github.com/srijanshetty/ShuffleRun} {Repo} }\\
        & \footnotesize{\textbf{OARS}: a music player which changes tracks on the basis of running speed of the user
            \href{https://github.com/srijanshetty/ShuffleRun} {Repo} }\\
        & \footnotesize{\textbf{Dotfiles}: an opinionated work flow on Linux Systems. \href{https://github.com/srijanshetty/dotfiles} {Repo} } \\
        & \footnotesize{\textbf{Prezto}: a fork of the Prezto ZSH framework  \href{https://github.com/srijanshetty/prezto} {Repo}} \\
        & \footnotesize {\textbf{oh-my-zsh}: a fork of the oh-my-zsh ZSH framework \href{https://github.com/srijanshetty/oh-my-zsh} {Repo} } \\
        & \footnotesize{\textbf{DS}: implementation of certain algorithms and data structures \href{https://github.com/srijanshetty/DS} {Repo}} \\

\end{tabular}

%----------------------------------------------------------------------------------------
%   SCHOLASTIC ACHIEVEMENTS
%----------------------------------------------------------------------------------------
\section{Scholastic Achievements}

\begin{tabular}{rp{12cm}}

    \textsc{2012-2013}   & \textbf{Academic Excellence Award}, IIT Kanpur. \\
    \textsc{2011-2012}   & \textbf{Academic Excellence Award}, IIT Kanpur. \\
    \textsc{Fall 2012}   & \textbf{Academic Mentor} for Y12 batch in \textbf{Fundamentals of Computer Science (ESC101)}. \\
                         & \footnotesize{As an academic Mentor, I was entrusted with the responsibility of making sure
                            that the students under me were able to keep pace with instructor and aid them in understanding
                            the difficult concepts of the course. }\\
    \textsc{Jan 2010}    & Qualified for \textbf{Indian National Mathematics Olympaid} (INMO) \\
    \textsc{May 2011}    & Secured All India rank \textbf{1937} in Joint Entrance Exam (JEE) \\
    \textsc{Sep 2010}    & \textbf{AIR 17} in \textbf{Technothlon} (Techniche) conducted by Indian Institute of Technology Guwahati. \\
    \textsc{2005, 2009}  & Awarded overall best student in grades 6th and 9th for displaying all round excellence. \\
    \textsc{2000 - 2009} & Recipient of scholastic excellence awards for exemplary academic performance throughout 2nd to 12th grade. \\

\end{tabular}

%----------------------------------------------------------------------------------------
%   SKILL SET
%----------------------------------------------------------------------------------------
\section{Skill Set}

\begin{tabular}{rl}

\textsc{Programming Languages}: %
    & \textbf{C}, \textbf{C++},\textbf{Python}, \textbf{JavaScript} \\
    \textsc{Web Development}: %
    & \textbf{HTML5}, \textbf{CSS}, \textbf{JavaScript}, \textbf{Flash}, \textbf{SQL}\\
    \textsc{Tools}: %
    & \textbf{Git}, \textbf{Shell Scripting}, \textbf{Prezi}, \textbf{Vim}, \textbf{LaTeX}\\
\\

\end{tabular}

%----------------------------------------------------------------------------------------
%   OTHER SKILLS
%----------------------------------------------------------------------------------------

\section {Social, Leadership and Artistic Skills}
\begin{tabular}{rp{12cm}}

    -    & Volunteered to help \textbf{Professor Bhaskar Dasgupta} teach English to underprivileged children from the campus
           and neighbouring village of Nankari.\\
    2014 & \textbf{Head, Major Events and Competitions, Antaragni'14} , the annual cultural festival of
    IIT Kanpur. \\
    2013 & \textbf{Hospitality Coordinator, Antaragni ‘13}\\
         & \footnotesize{As a Hospitality Coordinator, I was oversaw a team of 150 students responsible for
            the hospitality of the 1500 students who visit the campus for Antaragni, the annual cultural festival
            of IIT Kanpur which takes place in October every year.}\\
    2013 & \textbf{Editor-in-chief, Vox Populi}, the campus newsletter. \\
         & \footnotesize{Vox Populi is the campus newsletter of IIT Kanpur with a reach of more than 6000 students.
            As, the Editor-in-chief, I was responsible for ensuring impeccable quality of content in the newsletter and
            logistics of publishing the bi-monthly newsletter.}\\
    2012 & \textbf{Member Gymkhana Review Committee}\\
         & \footnotesize{The Gymkhana Review Committee was set up with the vision of revamping the Student's Gymkhana
            of IIT Kanpur, which is the body responsible for all extra curricular activities of the campus. As a
            member of the Gymkhana Review Committee, I chaired the sessions on \textbf{Extended Orientation of
            incoming freshmen students}, as a member of the committee, I gave valuable inputs to other academic,
            senate and activities reforms introduced by the committee.}\\
    2012 & \textbf{Elected Senator Students' Gymkhana} IITK Y11 batch. \\
    2012 & \textbf{Secretary, English Literary Society}. \\
    2010 & A year of schooling in \textbf{Indian Classical Music}. (Prayag Sangeet Samiti Allahabad). \\
    -   & Active blogger at \href{srijanshetty.quora.com} {srijanshetty.quora.com} \\

\end{tabular}

\end{document}
